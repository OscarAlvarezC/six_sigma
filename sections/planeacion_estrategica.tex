\section{Planeación estratégica}
\subsection{Antecedentes}
Al ser la planeación un pilar importante de la administración, sabemos que ha sido un proceso que ha estado presente en la organización del hombre prehispánico, pero no es hasta que se consolida en la jerga de los arquitectos antiguos que se le atribuye el nombre de planeación. Un concepto que nace de dibujar en papel (planos) las ideas que se concebían para que en un futuro se construyeran edificios, puentes, carreteras, etc. Con el pasar del tiempo, el concepto evolucionó de las obras civiles a cualquier trabajo que se pensara para el futuro, fuera éste una obra de arte, un cambio de estatus social, una transformación regional o una invasión territorial. Es decir, todo aquel planteamiento que demandaba tomar decisiones de forma anticipada y que se esperaba sucedería en un futuro, se le conocía como planeación. 


\subsection{Concepto}
Hoy en día hablamos de planeación cuando analizamos las consecuencias derivadas de una serie de decisiones que se requerirán tomarse en un futuro, con la esperanza de disipar la incertidumbre de lo que pudiera suceder. Esto nos lleva a pensar en el destino que buscamos y los medios para conseguirlo, así como los riesgos y oportunidades que el propio ambiente nos proporciona.
Ahora bien, otro concepto que se debe entender es el de estrategia, debido a que es parte fundamental del proceso. Estrategia viene del griego “strategos”, y las raíces de éste significan “ejército” y “acaudillar”, por lo que, si nos remontamos a la historia, este concepto ha sido usado por años en el ámbito militar. Hablando desde un punto empresarial, esta definición puede llegar a cuadrar, no porque las empresas proyecten destrucción en los competidores, sino porque están en competencia constante por vender más o tener mejores resultados que sus rivales.

Existen distintos puntos de vista, por ejemplo, Peter Drucker, en su libro “The Practice of Management” (1954), afirmaba que la estrategia requiere que los gerentes analicen su situación presente y que la cambien si es necesario. Parte de su definición partía de la idea que los gerentes deberían saber qué recursos tenía su empresa y cuáles debería tener.
Alfred Chandler definió estrategia empresarial, en su obra “Strategy and Structure” (1962), como el elemento que determinaba las metas básicas de una empresa, a largo plazo, así como la adopción de cursos de acción y la asignación de los recursos necesarios para alcanzar estas metas.
De acuerdo al concepto de planeación estratégica definido por el autor Steiner en su libro Planificación estratégica, lo que todo director debe saber- 1998, en donde define a la planeación estratégica formal considerando cuatro puntos de vista diferentes:
\begin{itemize}
	\item El porvenir de las decisiones actuales: la planeación trata con el porvenir de las decisiones actuales. Esto significa que la planeación estratégica observa la cadena de consecuencias de causas y efectos durante un tiempo, relacionada con una decisión real o intencionada que tomará el director. Si a este último no le agrada la perspectiva futura, la decisión puede cambiarse fácilmente. 
	\item Proceso: la planeación estratégica es un proceso que se inicia con el establecimiento de metas organizacionales, define estrategias y políticas para lograr estas metas, y desarrolla planes detallados para asegurar la implantación de las estrategias y así obtener los fines buscados”
	\item Filosofía: la planeación estratégica es una actitud, una forma de vida; requiere de dedicación para actuar con base en la observación del futuro, y una determinación para planear constante y sistemáticamente como una parte integral de la dirección.
	 \item Estructura: un sistema de planeación estratégica formal une tres tipos de planes fundamentales, que son: planes estratégicos, programas a mediano plazo, presupuestos a corto plazo y planes operativos.
\end{itemize}

Por su parte Torres (2014, p. 38) define el concepto de administración estratégica como: “Un proceso que explora y crea oportunidades nuevas y diferentes para el futuro de las organizaciones, basado en la planeación, en la implantación y ejecución de lo planeado y, en la evaluación de resultados con miras a tomar decisiones anticipadas, en cuanto a crecimiento, desarrollo, consolidación y cesación o desaparición”.

Torres (2014) enlista conceptos específicos de la administración estratégica tales como:
\begin{itemize}
	\item Giro o actividad, es decir la administración estratégica inicia con la definición de la actividad o negocio de la organización.
	\item Visión, proporciona un propósito intencionado para una orientación futura.
	\item Misión, identifica el alcance de las operaciones de una empresa en términos del producto y del negocio.
	\item Objetivos, engloba los resultados y logros que desean alcanzar las organizaciones para dar cumplimiento a su misión.
\end{itemize}


\subsection{Capacidad gerencial}
Por otra parte, otro concepto que vale la pena abordar dentro del proceso es la capacidad gerencial, clave en la planificación estratégica. Se denota como el conjunto de conocimientos, experiencias, habilidades, actitudes y aptitudes, que permite a las personas influir con medios no coercitivos sobre otras personas para alcanzar objetivos con efectividad, eficiencia y eficacia.

Los indicadores de gestión a tener en cuenta son:

\begin{itemize}
	\item Productividad: Relación entre los productos totales obtenidos y los recursos totales consumidos.
	\item Efectividad: Relación entre los resultados logrados y los que nos propusimos previamente y da cuenta del grado de cumplimiento de los objetivos planificados.
	\item Eficiencia: Relación entre la cantidad de recursos utilizados y la cantidad de recursos que se había estimado o programado utilizar.
	\item Eficacia: Valora el impacto de lo que hacemos, del producto que entregamos o del servicio que prestamos. No basta producir con 100% de efectividad, sino que los productos o servicios sean los adecuados para satisfacer las necesidades de los clientes.
\end{itemize}

\subsection{Proceso}
La planeación, al ser un proceso, está constituida por una base de diversas fases o etapas secuenciales, tales como:
\begin{itemize}
	\item Diagnóstico (descripción del medio ambiente interno y externo)
	\item Visión- Misión
	\item Objetivos
	\item Estrategias
	\item Presupuesto
	\item Evaluación de resultado
	\item Realimentación
\end{itemize}

Algo en lo que coinciden los autores que profundizan en el concepto de planeación, es que el proceso de planeación está constituido en términos generales de estructura en tres etapas:

\begin{enumerate}[I.]
	\item Etapa de la planeación. Se analiza el entorno; las oportunidades y posibles amenazas que el existen. Se define la serie de decisiones que serán tomadas y sus posibles consecuencias.
	\item Etapa de la ejecución o implantación. Es la puesta en marcha de las decisiones tomadas en la planeación.
	\item Etapa de control y evaluación. Una vez ejecutada la estrategia planteada, se realiza una evaluación de los resultados y se contrasta con lo ya esperado durante el proceso. A partir de la información recolectada y analizada en este proceso se tiene comienza con un proceso nuevo de análisis estratégico como cultura de una mejora continua.
\end{enumerate}

Estas etapas se ven representadas en el siguiente gráfico diseñado en 1997 por Johnson, G. y Scholes, K.

\begin{figure}[hbtp]
\centering
\includegraphics[scale=0.5]{images/diagrama_planificacion_estrategica.png}
\caption{Diagrama de elementos de la planificación estratégica}
\end{figure}

\subsection{Beneficios}
Un plan estratégico, trae beneficios relacionados con la gestión y la capacidad que se tiene para realizarla de manera eficiente, liberando recursos humanos y materiales, lo que nos lleva a la eficiencia productiva y una mejor calidad general para la organización.

Simplemente, establecer una visión, definir la misión, planificar y determinar objetivos, influye positivamente en el desempeño de la institución. La planificación estratégica permite pensar en el futuro, visualizar nuevas oportunidades y amenazas, así como, orientar de manera efectiva el rumbo de una organización, facilitando la acción de innovación de dirección y liderazgo.

La planificación estratégica es una manera intencional y coordinada de enfrentar la mayoría de los problemas críticos, intentando resolverlos en su conjunto y proporcionando un marco útil para afrontar decisiones, anticipando e identificando nuevas demandas.

Como nos indica Pimentel (1999), una buena planificación estratégica exige conocer más la organización, mejorar la comunicación y coordinación entre los distintos niveles y programas y mejorar las habilidades de administración. La planificación estratégica genera fuerzas de cambio que evitan que las organizaciones se dejen llevar por los cambios, las ayuda a tomar el control sobre sí mismas y no sólo a reaccionar frente a reglas y estímulos externos.

Por otro lado, Torres (2014) detalla los siguientes beneficios de acuerdo al enfoque, si se hablamos de rentabilidad, los beneficios pueden ser los siguientes:
\begin{enumerate}
	\item Factores externos:
		\begin{itemize}
			\item Regulación del gobierno
			\item Competencia
			\item Demanda del cliente
		\end{itemize}
	\item Capacidad e inventario:
		\begin{itemize}
			\item Planeación de la capacidad
			\item Inventario
			\item Compras
		\end{itemize}
	\item Producto
		\begin{itemize}
			\item Ingeniería de valor
			\item Diversidad de producto
			\item Investigación y desarrollo
		\end{itemize}
	\item Proceso
		\begin{itemize}
			\item Equipo
			\item Flujo del proceso
			\item Automatización
			\item Selección del proceso
		\end{itemize}
	\item Fuerza de trabajo
		\begin{itemize}
			\item Objetivos (Administración por objetivos)
			\item Sindicatos
			\item Remuneraciones
			\item Supervisión
			\item Estructura de la organización
			\item Diseño del trabajo
			\item Capacitación
			\item Selección y ubicación
		\end{itemize}
	\item Calidad
		\begin{itemize}
			\item Mejoramiento de la calidad
		\end{itemize}
\end{enumerate}