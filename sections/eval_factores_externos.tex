\section{Evaluación de factores externos}
\subsection{Entorno extorno}
Parte de la Planeación Estratégica es conocer el Entorno Externo de la organización, donde requiere aplicar todas las habilidades y conocimiento para llevar con éxito su ciclo de vida, por lo que es necesario identificar las amenazas y oportunidades en la organización.

La oportunidad en la organización la definimos como los factores, situaciones o actividades ajenas a la entidad, que de concretarse genere una ventaja competitiva sobre las demás organizaciones, por otro lado, cuando hablamos de Amenaza, lo definimos como los factores externos que, de concretarse afectarían la posición de la empresa. 

\subsection{Elementos del entorno}
El primer paso para hallar, pronosticar y evaluar los elementos influye en el macro ambiente de la organización, es a través del acróstico PEST-G, el cual hace referencia a los Sectores Políticos, Económicos, Sociales, Tecnológicos y Globales, tal como lo define la siguiente tabla diseñada en 2014 por Zacarías Torres Hernández. \newpage

\begin{longtable}{|m{8em}|m{12em}|m{12em}|}
 \hline 
 Fuerza del macro ambiente & Dimensión & Atiende \\ 
 \hline 
 Políticas, gubernamentales y legales & Comités de acción política. \newline
 Número de partidos políticos. \newline
 Estabilidad del gobierno.\newline
 Nivel de subsidios del gobierno.\newline
 Legislación sobre monopolios.\newline
 Legislación de Comercio exterior.\newline
 Leyes especiales, & Conductas, comportamientos y procedimientos de los gobiernos, puesto que los gobiernos son los reguladores, subsidiarios, patrones y clientes de todo tipo de organizaciones y personas \\ 
 \hline 
 Económicas & Ingreso per cápita. \newline
	Tasas de interés.\newline
	Precios del petróleo.\newline
	Tasa de inflación.\newline
	Saldos de las balanzas de divisas y comerciales.\newline
	Remesas del exterior.\newline
	Tasas y políticas fiscales.\newline
	Dependencia del exterior. & El curso y carácter de la economía donde la empresa se desempeña o podría hacerlo. Lo relacionado con los recursos, y tiene que ver con el mercado, las finanzas nacionales y la política monetaria. \\ 
 \hline 
 Sociales, culturales, demográficas y ambientales & Índice de educación.\newline
Número de divorcios.\newline
Mujeres en la población económicamente activa (PEA).\newline
Índices de natalidad y mortalidad.\newline
Tasas de emigración e inmigración.\newline
Libros leídos per cápita.\newline
Número de bibliotecas.\newline
Leyes de protección ambiental.\newline
Contaminación de agua y aire. & Actividades y valores sociales y culturales, dado que éstos son los pilares de toda organización social, muchas veces impulsan las condiciones y los cambios demográficos, económicos, políticos, legales y tecnológicos. Se incluyen actividades de ahorro, jubilación, trabajo, ocio, compras, moral y ética. \\ 
 \hline 
 Tecnológicas & Gasto público en investigación.\newline
Nuevos descubrimientos y desarrollos tecnológicos.\newline
Número de patentes.\newline
Número de investigadores por habitante.\newline
Regalías por asistencia técnica, patentes y marcas.\newline
Número de centros de investigación.\newline
Número de libros y artículos científicos publicados.\newline
Tasas de obsolescencia. & Cambios y descubrimientos tecnológicos revolucionarios que producen fuerte impacto. Adelantos de la superconductividad. Las instituciones y las actividades necesarias para crear conocimientos nuevos y convertirlos en información, productos, procesos y materiales nuevos. Algunos de estos productos tecnológicos son la manufactura integrada por computadora, la internet y la Tecnología de Alta Información (TAI). \\ 
 \hline 
 Globales & Hechos políticos importantes. \newline
Mercados globales críticos. \newline
Países recién industrializados. \newline
Distintos atributos de las culturas y las instituciones.\newline
Bloques económicos.\newline
Número de trabajadores en actividades de servicios.\newline
Comunicación en tiempo real.\newline
Transferencias de capitales. & Los nuevos mercados globales relevantes, los mercados existentes en proceso de cambio, los hechos políticos internacionales importantes y las características críticas de la cultura y las instituciones de los mercados globales. Las interacciones y acercamientos de información, ideas, capitales, personas, bienes y servicios.
 \\ 
 \hline 
 \caption{Elmentos de entorno (PEST-G).}
 \end{longtable}  
 
Este listado permite disponer de una idea realista de las oportunidades y amenazas que toda organización debe de atender para volverse más competitiva.

\subsection{Matriz de Evaluación de Factores Externos}
Para poder identificar este conjunto de factores se hace necesario el desarrollo de un proceso juicioso de auditoría externa que permita aproximar la realidad del entorno, todo esto apoyado con la herramienta de Evaluación de Factores Externos (EFE) o también conocida como la Matriz de Evaluación de los Factores Externos (MEFE).

La MEFE, es un instrumento de diagnóstico ponderado que permite hacer un estudio de las estrategias y evaluar las condiciones PEST-G, generando un listado de oportunidades reales que puede aprovechar la organización, así como un listado de amenazas las cuales servirán como guía para disminuir el riesgo.

La matriz EFE se desarrolla en 6 pasos principales.
\begin{enumerate}
	\item Elaboración de la lista de Oportunidades y Amenazas que afectan a la empresa y a su sector, donde uno debe ser lo más específico posible, apoyándose de porcentajes, índices y cifras comparativas.
	\item Se asigna a cada una de las entradas registrados un valor ponderado entre el 0.0 y 1.0, donde el 0 representa aquellas sin importancia y 1 las que son muy importantes. El valor indica la importancia relativa de dicho factor para tener éxito en el sector de la empresa. La suma de todos los valores asignados a los factores debe ser igual a 1.0.
	\item Se asigna una clasificación de uno a cuatro a cada factor externo, para identificar con cuánta eficacia se puede responder a la estrategia en ese momento, donde el cuatro representa una respuesta excelente, tres a la respuesta arriba del promedio, dos la respuesta es de nivel promedio y uno a la respuesta es deficiente.
	\item Se multiplica el valor de cada factor por su clasificación para determinar un valor ponderado.
	\item Se suma los valores ponderados de cada variable para determinar el valor ponderado total de la empresa.
	\item Se genera el enunciado o palabra clave del resultado, que asocie la oportunidad con la amenaza, de acuerdo con el mayor valor ponderado, generando una acción que mitigue o de solución a la amenaza.
\end{enumerate}

Ejemplo de la aplicación de la Matriz EFE para una empresa dedicada a realizar Cerveza Artesanal.

\begin{center}
[TABLA FACTORES EXTERNOS]
\end{center}

Un puntaje de valor ponderado total de 4.0 indica que una empresa responde de manera sorprendente a las oportunidades y amenazas presentes en su sector. Un puntaje total de 1.0 significa que las estrategias de la empresa no aprovechan las oportunidades ni evitan las amenazas externas.

Es importante observar que una comprensión minuciosa de los factores usados en la matriz EFE es más importante que las clasificaciones y los valores reales asignados.